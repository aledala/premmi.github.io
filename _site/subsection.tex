A function $f$ defined from $\R$ to $\R$ is an \textit{odd function} if $f(-x) = -f(x)$ for all $x$. Odd functions have the property that for any $a$, $$\int_{-a}^{a}f(x)\,dx = 0,$$ assuming that the integral exists.This is because the area under the function from
$-a$ to 0 cancels the area under the function from 0 to $a$.

To show this explicitly, let us integrate the first term in the sum from $-\infty$ to 0 and from 0 to $\infty$. This term should integrate to 0. 
\begin{align*}
& \,\,\,\int_{-\infty}^{\infty} \left(\frac{1}{2\pi\sigma^2}\right)^{1/2}\exp\left(-\frac{1}{2\sigma^2}y^2\right)y\,dy\\
&=\,\,\,\int_{-\infty}^{0} \left(\frac{1}{2\pi\sigma^2}\right)^{1/2}\exp\left(-\frac{1}{2\sigma^2}y^2\right)y\,dy\\ &\,\,\,\,\,\,\,\,+ \int_{0}^{\infty} \left(\frac{1}{2\pi\sigma^2}\right)^{1/2}\exp\left(-\frac{1}{2\sigma^2}y^2\right)y\,dy\\
&=\,\,\,\int_{\infty}^{0} \left(\frac{1}{2\pi\sigma^2}\right)^{1/2}\left(\frac{1}{2}\right)\exp\left(-\frac{1}{2\sigma^2}u\right)du\\ &\,\,\,\,\,\,\,\,+ \int_{0}^{\infty} \left(\frac{1}{2\pi\sigma^2}\right)^{1/2}\left(\frac{1}{2}\right)\exp\left(-\frac{1}{2\sigma^2}u\right)du \tag{i}\label{result}\\
&=\,\,\,\left(\frac{1}{2\pi\sigma^2}\right)^{1/2}\left(\frac{1}{2}\right)\left(-{2\sigma^2}\right)\\ &\,\,\,\,\,\,\,\,\,\,\left\lbrace\left[\exp\left(-\frac{1}{2\sigma^2}u\right)\right]_{\infty}^{0} + \left[\exp\left(-\frac{1}{2\sigma^2}u\right)\right]_{0}^{\infty}\right\rbrace\\
&=\,\,\,\left(\frac{1}{2\pi\sigma^2}\right)^{1/2}\left(\frac{1}{2}\right)\left(-{2\sigma^2}\right)\left\lbrace\left[\frac{1}{e^0}-\frac{1}{e^\infty}\right] + \left[\frac{1}{e^\infty}-\frac{1}{e^0}\right]\right\rbrace\\
&=\,\,\,0
\end{align*}
where we have used the change of variables $y^2=u$ and hence $2y\,dy=du$. Also from \eqref{result} we can immediately conclude that the expression integrates to 0 since $$\int_{a}^{b} f(x)\, dx + \int_{b}^{a} f(x)\, dx = \int_{a}^{b} f(x)\, dx - \int_{a}^{b} f(x)\, dx = 0$$