\documentclass[14pt]{extarticle}
\usepackage[margin=.75in, paperwidth=8.5in, paperheight=11in]{geometry}
\usepackage{amsthm,amsmath}
\usepackage{biblatex}
\title{Univariate Gaussian Distribution is Normalized}
\author{Premmi}
\date{\today}
\begin{document}
\maketitle	
\begin{flushleft}    
To prove that the univariate Gaussian distribution is normalized, we will first show that it is normalized for a zero-mean Gaussian and extend that result to show that $\mathcal{N}$(\textit{x}$\mid$$\mu$, $\sigma$$^{2}$) is normalized.		
\newline
\newline
The pdf of the zero-mean Gaussian distribution is given by:			
\begin{align}		
\varphi(x)= \frac{1}{\sqrt{2\pi\sigma^2}}\, \exp\left(-\frac{1}{2\sigma^2}x^2\right)\,\,\,\,\,\,\,-\infty<x<\infty.			
\end{align}
To prove that the above expression is normalized, we have to show that 
\begin{align}
\int_{-\infty}^{\infty} \exp\left(-\frac{1}{2\sigma^2}x^2\right)dx = \sqrt{2\pi\sigma^2}
\end{align}
\begin{proof}
Let
\begin{align}
I = \int_{-\infty}^{\infty} \exp\left(-\frac{1}{2\sigma^2}x^2\right)dx
\end{align}

Squaring the above expression,
\begin{align}
I^2 = \int_{-\infty}^{\infty}\int_{-\infty}^{\infty} \exp\left(-\frac{1}{2\sigma^2}x^2-\frac{1}{2\sigma^2}y^2\right)dx\,dy
\end{align}
To integrate this expression we make the transformation from Cartesian coordinates (x, y) to polar coordinates (r, $\theta$), which is defined by
\begin{align}
x = r \, cos\,\theta\\
y = r \, sin\, \theta
\end{align}
and using the trigonometric identity $cos^2\,\theta + sin^2\, \theta = 1$, we have $x^2 + y^2 = r^2$. Also the Jacobian of the change of variables is given by,
\begin{align*}
\dfrac{\partial \left(x, y\right)}{\partial \left(r, \theta\right)} \,\,&= \,\,\,\begin{vmatrix}
&\dfrac{\partial \left(x \right)}{\partial \left(r \right)} &\dfrac{\partial \left(x \right)}{\partial \left(\theta \right)}&\\
\\
&\dfrac{\partial \left(y \right)}{\partial \left(r \right)} &\dfrac{\partial \left(y \right)}{\partial \left(\theta \right)}&
\end{vmatrix}\\
\\
&= \,\,\,
\begin{vmatrix}
&cos \,\theta &-r\,sin \, \theta&
\\
&sin \,\theta &r\,cos\,\theta&
\end{vmatrix}\\
&=\,\,\, r\, cos^2\,\theta + r\, sin^2\,\theta
\\
&=\,\,\, r
\end{align*}
using the same trigonometric identity $cos^2\,\theta + sin^2\, \theta = 1$.
Thus equation (4) can be rewritten as 
\begin{align}
I^2 \,\,&=\,\,\, \int_{0}^{2\pi}\int_{0}^{\infty} \exp\left(-\frac{r^2}{2\sigma^2}\right) r\,dr\,d\theta 
\\
&= \,\,\,2\pi\int_{0}^{\infty} \exp\left(-\frac{r^2}{2\sigma^2}\right) r\,dr
\\
&= \,\,\,2\pi\int_{0}^{\infty} \exp\left(-\frac{u}{2\sigma^2}\right) \frac{1}{2}\,du
\\
&=\,\,\,\pi\left[\exp\left(-\frac{u}{2\sigma^2}\right)\,\left(-2\sigma^2\right)\right]_0^\infty
\\
&=\,\,\,2\pi\sigma^2
\end{align}
where we have used the change of variables $r^2 = u$.
Thus $$I = \left(2\pi\sigma^2\right)^{1/2}.$$
\\
Finally to prove that $\mathcal{N}$(\textit{x}$\mid$$\mu$, $\sigma$$^{2}$) is normalized, we make the tranformation $y = x - \mu$ so that,
\begin{align*}
\int_{-\infty}^{\infty} \mathcal{N}(\textit{x}\mid\mu, \sigma^{2})\,dx \,\,&= \,\,\,\frac{1}{\left(2\pi\sigma^2\right)^{1/2}} \int_{-\infty}^{\infty} \exp\left(-\frac{y^2}{2\sigma^2}\right)dy\\
&=\,\,\,\frac{I}{{\left(2\pi\sigma^2\right)^{1/2}}}\\
&=\,\,\,1
\end{align*} 
as required.
\end{proof}	
\end{flushleft}
\begin{thebibliography}{9}
\bibitem{kevinmurphy} 
Kevin P. Murphy. 
\textit{Machine Learning: A Probabilistic Perspective}. 
\textbf{Exercise 2.11} Normalization constant for a 1D Gaussian.
 
\bibitem{christopherbishop} 
Christopher M. Bishop. 
\textit{ Pattern Recognition and Machine Learning}.  
\textbf{Exercise 1.7}
\end{thebibliography}
\end{document}